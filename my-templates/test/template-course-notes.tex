\documentclass[18pt, openany]{book}
%%%%%%%%%%%%%%%%%%%%% Note Info. %%%%%%%%%%%%%%%%%%%

\newcommand{\notesname}{Notes of Course}
\newcommand{\authorname}{Author}
\newcommand{\authoremail}{author.name@example.com}
\newcommand{\authoruniversity}{University}

%%%%%%%%%%%%%%%%%%%%% Packages %%%%%%%%%%%%%%%%%%%%%

\usepackage{hyperref}
\usepackage{lipsum}
\usepackage{ctex}
\usepackage{framed}
\usepackage{setspace}
\usepackage{graphicx}
\usepackage{fancyhdr}
\usepackage{wrapfig}
\usepackage{enumitem}
\usepackage{float}
\usepackage{amsmath,amsthm,amsfonts,mathrsfs,amssymb,amscd}
\usepackage[bottom, hang,flushmargin]{footmisc}
\usepackage[utf8]{inputenc}
\usepackage[english]{babel}
\usepackage[dvipsnames]{xcolor}
\usepackage{tikz-cd}
\usepackage[most]{tcolorbox}
\usepackage{thmtools}
\tcbuselibrary{theorems}
\tcbuselibrary{breakable}

%%%%%%%%%%%%%%%%%%%%% Page Size / Head / Foot %%%%%%%%%%%%%%%%%%%%%
% \usepackage{showframe} % To render a frame marking the margins of a document
\usepackage[
    a4paper, 
    left=100pt, 
    right=100pt,
    % top=40pt,
]{geometry}
\marginparwidth = 0pt
% \headheight = 40pt
% \headsep = 15.0pt
% \textheight = 710pt

\setstretch{1.35} % the space between lines
\setlength{\parindent}{0pt} % the indent at the beginning of each paragraph
\setlength{\jot}{10pt}  % 设置每行之间的间距为 10pt

\pagestyle{fancy}
\fancyhf{} % clear the defalt style of page head and foot
\fancyhead[L]{\leftmark}
\fancyhead[R]{\rightmark}
\fancyfoot[C]{\thepage}

\newcommand{\HRule}[1]{\rule{\linewidth}{#1}}

%%%%%%%%%%%%%%%%%%%%% Math Environments %%%%%%%%%%%%%%%%%%%%%

\declaretheorem[numberwithin=section]{definition}
\declaretheorem[numberwithin=section]{theorem}
\declaretheorem[sibling=theorem]{proposition}
\declaretheorem[sibling=theorem]{lemma}
\declaretheorem[sibling=theorem]{corollary}
\declaretheorem[numberwithin=section]{remark}
\declaretheorem[numberwithin=section]{example}
\declaretheorem[numberwithin=section]{exercise}

\tcolorboxenvironment{definition} {colback=white, boxrule=0.5pt, sharp corners, breakable}
\tcolorboxenvironment{theorem}    {colback=white, boxrule=0.5pt, sharp corners, breakable}
\tcolorboxenvironment{proposition}{colback=white, boxrule=0.5pt, sharp corners, breakable}
\tcolorboxenvironment{lemma}      {colback=white, boxrule=0.5pt, sharp corners, breakable}
\tcolorboxenvironment{corollary}  {colback=white, boxrule=0.5pt, sharp corners, breakable}
\tcolorboxenvironment{remark}     {colback=white, boxrule=0.5pt, sharp corners, breakable}
\tcolorboxenvironment{example}    {colback=white, boxrule=0.5pt, sharp corners, breakable}
\tcolorboxenvironment{exercise}   {colback=white, boxrule=0.5pt, sharp corners, breakable}

\renewcommand\qedsymbol{$\blacksquare$}
\newenvironment{solution}{{\nointent\it Solution.\hspace{-\baselineskip} } }{\hfill$\blacksquare$\par}
\newcommand{\defi}[2][]{\begin{definition} [\textbf{#1}] \normalfont #2 \end{definition}}
\newcommand{\thm} [2][]{\begin{theorem}    [\textbf{#1}] \normalfont #2 \end{theorem}}
\newcommand{\prop}[2][]{\begin{proposition}[\textbf{#1}] \normalfont #2 \end{proposition}}
\newcommand{\lem} [2][]{\begin{lemma}      [\textbf{#1}] \normalfont #2 \end{lemma}}
\newcommand{\cor} [2][]{\begin{corollary}  [\textbf{#1}] \normalfont #2 \end{corollary}}
\newcommand{\rmk} [2][]{\begin{remark}     [\textbf{#1}] \normalfont #2 \end{remark}}
\newcommand{\expl}[2][]{\begin{example}    [\textbf{#1}] \normalfont #2 \end{example}}
\newcommand{\exrc}[2][]{\begin{exercise}   [\textbf{#1}] \normalfont #2 \end{exercise}}
\newcommand{\prof}[1]{\begin{proof} #1 \end{proof}}
\newcommand{\soln}[1]{\begin{solution} #1 \end{solution} }
%\renewcommand{\thesection}{\arabic{section}}  % 将 section 的编号设为 1, 2, 3 ...
%\renewcommand{\thesubsection}{\thesection}    % 将 subsection 的编号设为与 section 相同
%\numberwithin{theorem}{section}               % 使定理编号跟随 section 重置

%%%%%%%%%%%%%%%%%%%%%%%%%%%%%%%%%%%%%%%%%%%%%%%%%%%%%%%%%%%%%%%%%%%%%%%%%%%%%%%%
%%%%%%%%%%%%%%%%%%%%%%%%%%%%%%%%%%% Document %%%%%%%%%%%%%%%%%%%%%%%%%%%%%%%%%%%
%%%%%%%%%%%%%%%%%%%%%%%%%%%%%%%%%%%%%%%%%%%%%%%%%%%%%%%%%%%%%%%%%%%%%%%%%%%%%%%%

\begin{document}

%%%%%%%%%%%%%%%%%%%%% Title / Contents %%%%%%%%%%%%%%

\title{ \HRule{1.5pt} \\ [15.0pt] \LARGE \textbf{\notesname} \HRule{1.5pt} }
\author{ 
    \textbf{\authorname} \\ 
    \authoruniversity \\ 
    \href{mailto:\authoremail}{\texttt{\authoremail}}
}
\date{\textbf{Notes Date}}
\maketitle
\newpage
\tableofcontents
\newpage

\setcounter{page}{1}

% Bibliography --------------------------------------

\begin{thebibliography}{99}
    \bibitem{1} Author Name. \textit{Title of the Book}. Publisher, Year.
\end{thebibliography}

%%%%%%%%%%%%%%%%%%%%% Chapter 1 %%%%%%%%%%%%%%%%%%%%%
\chapter{First Chapter Name}

%%%%%%%%%%%%%%%%%%%%% Section 1 %%%%%%%%%%%%%%%%%%%%%
\section{First Section Name}

\subsection{All Math Environments}

\defi[Definition Name]{
    Definition Explanation from \cite{1}.
}

\thm[Theorem Name]{ \label{good_thm}
    Theorem Explanation.
}

\prof{
    Here we write\footnote{This is a footnote.} our proof.
}

\prop[Proposition Name]{
    Proposition Explanation.
}

\lem[Lemma Name]{
Lemma Explanation.
\begin{itemize}
    \item item one
    \item item two
\end{itemize}
}

\cor[Corollary Name]{
    Corollary Explanation.
}

\rmk[Remark Name]{
    Remark Explanation.
}

\expl[Example Name]{
    Example Explanation.
}

% \soln{
%     Here we write our solution.
% }

\exrc[Exercise Name]{
    Exercise Explanation.
}

%%%%%%%%%%%%%%%%%%%%% Section 2 %%%%%%%%%%%%%%%%%%%%%
\section{Second Section Name}

\lipsum[1-2]


%%%%%%%%%%%%%%%%%%%%% Section 3 %%%%%%%%%%%%%%%%%%%%%
\section{Third Section Name}

\defi[Definition Name]{
    Definition Explanation.
}

\thm[Theorem Name]{
    Theorem Explanation.
}

\prof{
    Here\footnote{This is another footnote.} we write our proof.
    We can refer a good theorem \hyperref[good_thm]{good thm 1.1.1}
}

\prop[Proposition Name]{
    Proposition Explanation.
}

\lem[Lemma Name]{
Lemma Explanation.
\begin{itemize}
    \item item one
    \item item two
\end{itemize}
}

\cor[Corollary Name]{
    Corollary Explanation.
}

\rmk[Remark Name]{
    Remark Explanation.
}

\expl[Example Name]{
    Example Explanation.
}

% \soln{
%     Here we write our solution\footnote{
%         Here we have ananother footnote.
%     }.
% }

\exrc[Exercise Name]{
    Exercise Explanation.
}

%%%%%%%%%%%%%%%%%%%%%%%%%%%%%%%%%%%%%%%%%%%%%%%%%%%%%

\end{document}

%%%%%%%%%%%%%%%%%%%%%%%%%%%%%%%%%%%%%%%%%%%%%%%%%%%%%%%%%%%%%%%%%%%%%%%%%%%%%%%%
%%%%%%%%%%%%%%%%%%%%%%%%%%%%%%% Document End %%%%%%%%%%%%%%%%%%%%%%%%%%%%%%%%%%%
%%%%%%%%%%%%%%%%%%%%%%%%%%%%%%%%%%%%%%%%%%%%%%%%%%%%%%%%%%%%%%%%%%%%%%%%%%%%%%%%